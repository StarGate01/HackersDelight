\section{Motivation and Relevance}

Digital wireless communication has become irreplaceable in our modern information technology and society. Communication is done primarily “on the go” using smartphones instead of cable bound landlines. The trusty Ethernet-port is disappearing in modern ultra-slim notebook computers, instead networks are accessed solely via Wi-Fi (IEEE 802.11). Even some household devices such as fridges or even dishwashers can be connected to the internet via Wi-Fi.

Developments like these do no only lead to a world full of exciting new possibilities, but also to a tremendous amount of new potential security risks. Hackers can and do use the ever-growing amount of unsecure or badly secured wireless networks for their own illicit purposes, such as ordering illegal goods or obscuring their access point, or just surfing the internet for free without paying a service provider to access to their infrastructure.  

The reasons to break into a secured wireless network however do not necessarily have to be illegitimate. In fact, the best way to verify a security concept and implementation is to attack it in a controlled and documented manner --- the art of "Penetration testing".

\section{Testing setup}

To avoid legal issues, a controlled testing setup was constructed for the purpose of this research:

\begin{itemize}

\item{Lenovo Y50-70 \textbf{Laptop} running Kali Linux \cite{Kali17}}

\item{TP-LINK TL-WN722N \textbf{USB-WiFi antenna}}

\item{TP-LINK TL-WR542G \textbf{wireless router}}

\item{Samsung ATIV-S \textbf{Smartphone}}

\end{itemize}

A USB antenna was chosen for its extended range, compared to the laptops' integrated Wi-Fi antenna.

The operation system "Kali Linux" was chosen for its flexibility and vast collection of pre-installed penetration testing software.

Any other devices were already owned and available.

\section{Attacking networks}

What is relevant today

\subsection{Attacking WEP-secured networks}

\subsection{Attacking WPA2-secured networks}

\subsection{Attacking the user}

\section{Conclusion}